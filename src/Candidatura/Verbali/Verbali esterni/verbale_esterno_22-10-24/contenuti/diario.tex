\section{Diario della riunione}

\begin{itemize}
    \item Discussione con il proponente \emph{Ergon} sull'artefatto richiesto, guidata dalle nostre domande.
    \item Risposta del proponente ad alcune domande:
\end{itemize}
\vspace{1cm}
\begingroup
\renewcommand{\ni}{\noindent}

{\Large Domande funzionali}

\vspace{0.5cm}

\begin{tabular}{>{\justifying\arraybackslash}p{0.4\textwidth} >{\justifying\arraybackslash}p{0.6\textwidth}}
    \multicolumn{1}{c}{\textbf{Domande}} & \multicolumn{1}{c}{\textbf{Risposte}} \\ \\
    \ni Bisogna costruire il modello di embedding da zero oppure è possibile sfruttare langchain? Nel caso quali tecnologie sono consigliate? & \ni Si possono usare anche cose già fatte, integrandole nel progetto. Libertà piena \\ \\
    \ni Cosa intendete per template? nello specifico la configurazione dei template delle domande nel back-end deve essere aperta alla modifica in caso diventassero antiquati? & \ni Ci sono due possibili tipologie di domande: template di domanda e risposta, e la totale IA generativa. Se ci sono feedback, l'obbiettivo è andare mano mano a creare dei template di domande frequenti a risposta precisa. \\ \\
    \ni Come dovrà essere fatta l'interfaccia utente per la configurazione della piattaforma? & \ni Un'unica chat. Interessante la cronologia, ma non essenziale. \\ \\
    \ni Come viene raccolto e utilizzato il feedback dell'utente per migliorare le risposte future nel sistema? & \ni Bisogna valutare in base all'LLM preaddestrato da cui attingeremo. In fase iniziale, va bene anche solo il pollice in su e in giù. In fase avanzata, il chatbot può anche chiedere all'utente che cosa si aspettava \\ \\
    \ni Come deve essere visualizzato il prompt e strutturato l’output? deve essere anche desktop o solo mobile? & \ni  Idea era come un'app mobile, preferenza per .NET MAUI. Sono aperti anche ad una web app. Focus comunque sul mobile \\ \\
\end{tabular}
\begin{tabular}{>{\justifying\arraybackslash}p{0.4\textwidth} >{\justifying\arraybackslash}p{0.6\textwidth}}
    \ni Quali sono le funzionalità richieste lato front-end? deve essere solamente chat testuale o deve permettere il caricamento di un file? & \ni Semplicemente chat \\ \\
    \ni Come deve essere gestito il login? dovrà avere un'utenza singola o multipla? & \ni Un account per membro di azienda, quindi multipli \\ \\
    \ni È richiesto di memorizzare le chat con il bot? se si, quante interazioni? & \ni Non è obbligatorio salvare uno storico \\ \\

\end{tabular}

\vspace{1.5cm}

{\Large Domande non funzionali}

\vspace{0.5cm}

\begin{tabular}{>{\justifying\arraybackslash}p{0.4\textwidth} >{\justifying\arraybackslash}p{0.6\textwidth}}
    \multicolumn{1}{c}{\textbf{Domande}} & \multicolumn{1}{c}{\textbf{Risposte}} \\ \\
    \ni Bisogna limitare gli argomenti che si possono trattare con il sistema? & \ni Loro forniranno un contesto. Ideale sarebbe limitarlo lì, non lasciarlo troppo libero \\ \\
    \ni Ci potreste descrivere i dati aziendali che dobbiamo utilizzare ed il modo a cui accedervi? & \ni Sì, verrà fornito un accesso in futuro \\ \\
    \ni Ci potete fornire dei comportamenti tipo in modo da istruire il bot?(nello specifico, definiamo delle regole che il bot deve seguire) & \ni Sì, probabilmente mondo del beverage, aziende che distrubuiscono bevande \\ \\
    \ni Ci potreste dare APIkey per i vostri strumenti? & \ni Forniranno aiuto anche con strumenti aziendali. Mentre i dati vengono forniti tramite CSV \\ \\
\end{tabular}

\vspace{1.5cm}

{\Large Altro}
\nopagebreak
\vspace{0.5cm}

\begin{tabular}{>{\justifying\arraybackslash}p{0.4\textwidth} >{\justifying\arraybackslash}p{0.6\textwidth}}
    \multicolumn{1}{c}{\textbf{Domande}} & \multicolumn{1}{c}{\textbf{Risposte}} \\ \\
    \ni Quali sono le tempistiche per i corsi online che ci mettete a disposizione? & \ni I corsi sono offline, li possiamo consultare quando vogliamo. Ce ne sarà uno su .NET MAUI e anche su LLM. Tutta documentazione offline \\ \\
    \ni Possiamo utilizzare qualsiasi LLM Esterno? & \ni Possiamo scegliere quello che vogliamo. Anche non open source, ma lo dobbiamo chiedere. \\ \\
\end{tabular}


\endgroup