\section{Diario della riunione}

\begin{itemize}
    \item Verificati e approvati i documenti per la candidatura all'appalto, per i quali rimagono le seguenti ultime modifiche da apportare:
    \begin{itemize}
      \renewcommand{\labelitemii}{--}
      \item Per la Valutazione dei Capitolati, rimangono da ricopiare gli ultimi contenuti in LaTeX, dunque è stato
            incaricato Righetto Filippo di completare il lavoro inserendo la valutazione di \emph{Vimar GenIAle}.
      \item Per la Lettera di Presentazione, il gruppo ha deciso anche di segnalare il contatto via mail che abbiamo intrattento con \emph{AzzurroDigitale}, 
            oltre al già segnalato contatto telematico. Viene dunque incaricato Verzotto Davide di aggiungere la riga in più con la domanda e la risposta
            che ci siamo scambiati via mail.
      \item Per il Preventivo dei Costi ed Impegni Orari, il gruppo ha approvato la data di termine del progetto fissandola al \emph{22/04/24}, 
            ma ha deciso di diminuire il monte ore da dedicare al progetto per ciascun membro, spostandolo a 92 ore. Viene dunque incaricata Federica 
            Bolognini di apportare le modifiche necessarie alla suddivisione dei ruoli e al costo totale dell'impegno.
    \end{itemize}
    \item E' stato valutato che iniziare già a redigere la documentazione richiesta per la Requirements and Technology Baseline sarebbe un anticipo troppo
          ampio, ed è stata dunque rimandata la questione a dopo l'assegnazione dell'appalto di lunedì \emph{4 novembre 2024}. Tuttavia, il gruppo si riserva
          di indagare ed informarsi preventivamente sul tema.
    \item Per la lezione rovesciata di mercoledì \emph{6 Novembre 2024} a tema Documentazione, è stato deciso di preparare tutti quanti delle domande, e, 
          se ci si dovesse accorgere che individualmente ne sono state pensate poche, il gruppo si unirà e parteciperà collettivamente alla lezione.
\end{itemize}



