\section{Valutazione degli altri Capitolati}


\subsection{C1: Artificial QI}

\subsubsection{Pro}

\begin{itemize}
    \item \textbf{Innovazione e utilità}: il gruppo trova interessante cercare di trovare una soluzione per il problema molto attuale della valutazione delle abilità degli LLM.
    \item \textbf{Punti di partenza}: il proponente Zucchetti è disponibile a fornire un programma esterno che espone delle API da testare, che è un'opportunità per avere una parte del sistema già pronta.
    \item \textbf{Molti requisiti opzionali}: sono consigliati vari spunti per estendere l'applicazione nel caso in cui si riesca a concludere in tempi brevi i requisiti obbligatori.
\end{itemize}

\subsubsection{Contro}

\begin{itemize}
    \item \textbf{Complessità del problema}: il capitolato tratta un problema aperto nel mondo dell'informatica, il che costituisce una complessità di ricerca esplorativa che va oltre lo sviluppo dell'applicativo e che non è calcolabile a priori.
    \item \textbf{Genericità degli strumenti}: vengono presentate molte poche tecnologie consigliate rispetto ad altri capitolati, e i suggerimenti rimangono molto generici a riguardo.
    \item \textbf{Dominio incognito e dataset assente}: non è chiaro quale sia il dominio di conoscenza che è da testare nell'LLM di turno, infatti non viene mai specificato che tipo di domande e risposte devono essere presenti nel dataset. Sembra inoltre che sia compito del gruppo scrivere le domande e risposte che facciano da training set, il che costituisce un onere non da poco considerata l'enorme quantità di dati necessari per allenare il sistema.
\end{itemize}


\subsection{C3:	Automatizzare le routine digitali tramite l’intelligenza generativa}

\subsubsection{Pro}

\begin{itemize}
    \item \textbf{Automazione delle Routine}: È interessante che il progetto punti ad automatizzare compiti ripetitivi come l’organizzazione delle e-mail e la gestione dei calendari, migliorando così l’efficienza e l'organizzazione aziendale.
    \item \textbf{Estendibilità}: L’applicativo è concepito in modo modulare, permettendo di aggiungere facilmente nuove automazioni e blocchi funzionali.
    \item \textbf{Supporto e Formazione}: Var Group fornisce supporto continuo per le tecnologie utilizzate, formazione e incontri periodici per garantire la qualità del progetto e per gestire l'evoluzione del software.
\end{itemize}

\subsubsection{Contro}

\begin{itemize}
    \item \textbf{Incremento dei tempi di sviluppo}: Potremmo dover affrontare ritardi dovuti alla necessità di studiare nuove e diverse tecnologie tra cui Python, Swift, NodeJS, MongoDB e Typescript. Questo potrebbe impattare la consegna nei tempi previsti, con conseguente aumento dei costi di progetto.
    \item \textbf{Scarso interesse}: il gruppo ha preferito concentrarsi su altri progetti con focus più accattivanti.
\end{itemize}


\subsection{C4:	NearYou - Smart custom advertising platform}

\subsubsection{Pro}

\begin{itemize}
    \item \textbf{Innovatività}: il progetto utilizza tecnologie avanzate come intelligenza artificiale generativa, data stream processing e targeting geografico in tempo reale. Questo lo rende un progetto ideale per acquisire esperienza con tecnologie molto richieste nel mercato attuale.
    \item \textbf{Flessibilità tecnologica}: il progetto lascia ampia libertà di scelta sugli strumenti e tecnologie da adottare.
    \item \textbf{Licenza}: la licenza del software sarà interamente nostra, potendo esporre il progetto nel curriculum.
\end{itemize}

\subsubsection{Contro}

\begin{itemize}
    \item \textbf{Complessità elevata}: il progetto richiede la gestione di argomenti forse troppo complessi data la scarsa esperienza dei membri del gruppo negli ambiti richiesti.
    \item \textbf{Rischio di dispersione}: la possibilità di scegliere le tecnologie da utilizzare potrebbe essere un'arma a doppio taglio, portando a dispersione o a scelte tecniche inadeguate, soprattutto se non si è esperti in tutti gli ambiti.
    \item \textbf{Scarso interesse}: il gruppo ha preferito concentrarsi su altri progetti con focus più accattivanti.
\end{itemize}


\subsection{C5:	3Dataviz}

\subsubsection{Pro}

\begin{itemize}
    \item \textbf{Gestione dei dati}: Il progetto è molto utile per quanto riguarda l’apprendimento, la gestione e la visualizzazione di un grande numero di dati in maniera chiara e interattiva.
    \item \textbf{Ampia libertà progettuale}: Il progetto offre grande libertà dal punto di vista della realizzazione dei vari grafici e delle funzionalità richieste. Questa flessibilità consente di personalizzare il sito in base alle esigenze e alle competenze del team.
\end{itemize}

\subsubsection{Contro}

\begin{itemize}
    \item \textbf{Richieste competenze avanzate}: La creazione e gestione di un sistema di visualizzazione 3D richiede delle competenze tecniche molto avanzate nel campo della grafica tridimensionale.
    \item \textbf{Studio di tecnologie nuove}: Il progetto richiede lo studio di tecnologie mai utilizzate prima, rendendo necessario un approfondimento e uno studio individuale per acquisire le competenze necessarie.
    \item \textbf{Scarso interesse}: il gruppo ha preferito concentrarsi su altri progetti con focus più accattivanti.
\end{itemize}


\subsection{C6: Sistema di gestione di un magazzino distribuito}

\subsubsection{Pro}

\begin{itemize}
    \item \textbf{Machine Learning}: Si richiede l'utilizzo del machine Learning per prevedere la richiesta futura di un certo materiale, materia che può essere molto interessante su cui lavorare.
    \item \textbf{Utilità}: Il problema della gestione del magazzino c’e’ da molti anni e colpisce molte aziende, imparare a creare software innovativi per la gestione di esso può essere molto utile e formativo.
\end{itemize}

\subsubsection{Contro}

\begin{itemize}
    \item \textbf{Complessità dei microservizi}: L'architettura basata su microservizi introduce maggiore complessità nella gestione del sistema e nell'orchestrazione, richiedendo più competenze e lavoro..
    \item \textbf{Sicurezza}: Richiesta di sicurezza avanzata tra il sistema centrale e i sistemi locali, non richiesto da altri capitolati. Questa richiesta aumenta la complessità del sistema, rendendo anche più difficile quantificare le ore di allenamento necessarie.
    \item \textbf{Requisiti di test e validazione}: La necessità di effettuare test estensivi e validazioni di algoritmi predittivi e di sicurezza, implica un elevato sforzo continuo.
\end{itemize}


\subsection{C8: Requirement Tracker - Plug-in VS Code}

\subsubsection{Pro}

\begin{itemize}
    \item \textbf{Supervisione continua}: disponibilità dell’azienda ad essere presente sia in presenza che in remoto, potendo così assistere il gruppo e guidarlo nella realizzazione.
    \item \textbf{Modelli di AI pre-addestrati}: è possibile utilizzare modelli di AI già esistenti e sfruttarli tramite API. Il team potrebbe quindi iniziare a lavorare senza dover programmare da zero un modello.
\end{itemize}

\subsubsection{Contro}

\begin{itemize}
    \item \textbf{Necessario studio di tecnologie sconosciute}: è necessario implementare un’architettura di tipo modulare, che il gruppo non ha ancora usato approfonditamente, dunque nonostante l’occasione, il suo apprendimento rallenterebbe il lavoro per il tempo necessario.
    \item \textbf{Vincoli stringenti}: i vincoli obbligatori imposti sono in quantità tale da presentare un ridotto spazio di manovra o di inventiva per il team.
    \item \textbf{Proof of Concept}: la necessità di sviluppare un P.o.C. già a metà progetto, può richiedere uno sforzo elevato, soprattutto essendo previste numerose tecnologie nuove da apprendere.
\end{itemize}