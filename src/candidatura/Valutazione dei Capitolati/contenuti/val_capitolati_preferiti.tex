\section{Valutazione dei Capitolati preferiti}


\subsection{C7: LLM: Assistente digitale}

\subsubsection{Descrizione}
Il Progetto prevede la realizzazione di un assistente virtuale 
che aziende il cui core business è dato dalla vendita di prodotti 
possono mettere a disposizione dei propri clienti, 
al fine di facilitare la ricerca di informazioni sui prodotti disponibili 
e rispondere alle domande più frequenti.
\subsubsection{Dominio}
\paragraph{Dominio tecnologico:}
Non ci sono tecnologie obbligatorie, ma sono state consigliate:
\begin{itemize}
    \item \textit{\textbf{MySQL}}: per la gestione del database relazionale.
    \item \textit{\textbf{LLM}}: il proponente ha indicato un insieme di modelli open source, tra questi si evidenzia Italia by iGenius, modello con 9 miliardi di parametri addestrato con un dataset al 90\% italiano.
    \item \textit{\textbf{API Rest}}: per la comunicazione tra il modello LLM e l'applicativo di interazione con l'utente.
    \item \textit{\textbf{ODBC}}: Open Database Connectivity, standard per la comunicazione da e per il database.
    \item \textit{\textbf{.NET MAUI}}: framework per lo sviluppo di applicazioni cross-platform.
\end{itemize}
\paragraph{Dominio applicativo:}
Il progetto si inserisce nel contesto della digitalizzazione e dell’uso del Machine Learning 
per analizzare grandi quantità di dati aziendali e sviluppare sistemi di interazione uomo-macchina avanzati. 
Nello specifico è rivolto alle aziende del settore della vendita di alimenti, 
dove la conoscenza dettagliata dei prodotti è spesso affidata a specialisti. 
Mira a creare un assistente virtuale che aiuti i clienti a trovare informazioni sui prodotti disponibili 
e risponda alle domande frequenti, migliorando l’accessibilità delle informazioni 
e l’efficienza del processo di assistenza clienti, riducendo la dipendenza dagli specialisti umani 
e offrendo risposte rapide e accurate alle domande dei clienti.
\subsubsection{Criticità riscontrate}
Fatta eccezione per l’elevata complessità delle tecnologie necessarie per lo sviluppo, 
che richiederà un certo periodo di studio da parte di tutti i membri del gruppo, 
non abbiamo riscontrato importanti criticità. 
Tuttavia ci sono alcuni aspetti che ci hanno convinto a scegliere il capitolato C9 piuttosto di questo. 
I due capitolati trattano progetti molto simili, ma riteniamo che per quanto riguarda il capitolato C9 
l’azienda proponente offra un supporto più adeguato alle nostre esigenze, 
e le tecnologie suggerite hanno suscitato maggiore interesse da parte del gruppo.
\subsubsection{Conclusioni}
Il gruppo ha apprezzato molto il progetto presentato, 
riconoscendone il valore e la potenziale utilità. 
Tuttavia, abbiamo deciso di non procedere con questo progetto, 
non per una mancanza di qualità, ma perché un altro capitolato ha suscitato un maggiore interesse da parte nostra.

\subsection{C2: Vimar GENIALE}

\subsubsection{Descrizione}

\subsubsection{Dominio}

\subsubsection{Criticità riscontrate}

\subsubsection{Conclusioni}