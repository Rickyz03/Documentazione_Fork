\section{Valutazione del Capitolato selezionato}

\subsection{C9: BuddyBot}

\subsubsection{Descrizione}
Il capitolato richiede di sviluppare un LLM capace di rispondere
a diversi tipi di richieste relative ai dati interni all'azienda, 
rendendo inoltre possibile per gli utenti ricevere informazioni riguardanti 
i contenuti presenti su piattaforme esterne.

\subsubsection{Dominio}
\paragraph{Dominio tecnologico:}
Viene richiesto l'utilizzo di un LLM open-source o di API di modelli 
a pagamento (come ChatGPT). Il modello deve poter comunicare tramite le API di Jira, 
Confluence e GitHub. Una volta recuperate le informazioni necessarie per la risposta,
deve elaborarle e formulare una risposta finale. Tutte le chat devono essere
salvate su un database e devono poter essere recuperate. L'azienda non richiede
l'uso di una tecnologia specifica, ma fornisce alcuni suggerimenti:
\begin{itemize}
    \item \textit{\textbf{OpenAI}}: Utilizzo tramite API per le funzionalità NLP, ovvero di comprensione e generazione del testo.
    \item \textit{\textbf{LangChain}}: Progetto open-source che permette l'integrazione di modelli AI senza richiedere una conoscenza dettagliata del loro funzionamento interno.
    \item \textit{\textbf{Angular}}: Framework front-end per la costruzione di applicazioni web.
    \item \textit{\textbf{Node/NestJS}}: Framework per lo sviluppo di applicazioni server-side orientato ai microservizi e alle API RESTful.
    \item \textit{\textbf{Spring Boot}}: Framework Java che offre strumenti per la creazione di applicazioni standalone, con supporto integrato per database, sicurezza e gestione delle dipendenze.
\end{itemize}


\paragraph{Dominio applicativo:}
BuddyBot vuole diventare un assistente a 360 gradi per qualsiasi membro 
dell'azienda. Si propone di essere facilmente accessibile tramite
un'interfaccia web, dove l'utente può formulare domande e recuperare lo storico delle vecchie chat.
Il bot deve poter rispondere a domande riguardanti file presenti in Jira, GitHub e Confluence, 
riuscendo anche a incrociare dati provenienti da diverse fonti. Il compito principale di 
BuddyBot sarà quindi aiutare i membri dell'azienda a trovare e combinare i dati in 
modo più rapido e preciso, fornendo risposte in linguaggio naturale.

\subsubsection{Motivazione della scelta}
La scelta finale è ricaduta su questo capitolato per diversi motivi:
\begin{itemize}
    \item \textit{\textbf{Disponibilità dell'azienda}}: L'azienda si è dimostrata da subito molto disponibile; le risposte alle mail sono arrivate in brevissimo tempo e si è subito manifestata la volontà di organizzare un incontro. Inoltre, si sono proposti di fornire anche supporto tecnico durante lo sviluppo, qualora ce ne fosse bisogno.
    \item \textit{\textbf{Interesse}}: Il prodotto finale ha suscitato grande interesse nel gruppo, soprattutto per le tecnologie con cui prevediamo di lavorare per realizzarlo. In particolare, ci è sembrato il capitolato con il miglior utilizzo dell'IA.
    \item \textit{\textbf{Fattibilità del progetto}}: Confrontando le richieste tra i vari capitolati, BuddyBot ci è sembrato il miglior compromesso tra difficoltà e interesse.
    \item \textit{\textbf{Libertà tecnologica}}: La libertà lasciata nella scelta delle tecnologie ci è sembrata molto vantaggiosa, permettendoci di utilizzare tecnologie con le quali alcuni membri del gruppo hanno già esperienza.
\end{itemize}

\subsubsection{Conclusioni}
Visti i motivi sopra citati e non trovando in nessun altro capitolato una combinazione equivalente, 
il gruppo ha deciso di scegliere BuddyBot come progetto per cui candidarsi.
