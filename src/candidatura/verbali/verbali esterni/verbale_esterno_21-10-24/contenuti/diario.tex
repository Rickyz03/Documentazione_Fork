\section{Diario della riunione}

\begin{itemize}
    \item Discussione con il proponente \emph{AzzurroDigitale} sull'artefatto richiesto, guidata dalle nostre domande
    \item Risposta del proponente ad alcune domande:
\end{itemize}

\vspace{0.5cm}

\begingroup
\renewcommand{\ni}{\noindent}

  {\Large Domande funzionali}

  \vspace{0.5cm}

  \begin{tabular}{>{\justifying\arraybackslash}p{0.5\textwidth} >{\justifying\arraybackslash}p{0.6\textwidth}}
      \multicolumn{1}{c}{\textbf{Domande}} & \multicolumn{1}{c}{\textbf{Risposte}} \\ \\
      
      \ni 1. Come deve essere strutturata la parte web? (nello specifico: deve essere disponibile anche una versione mobile-friendly? il sito deve essere integrato in un sito vostro già esistente?l'interazione sarà solo testuale oppure deve permettere l'iintegrazione di caricamento di un file?)
      & \ni Frontend deve essere un’interfaccia in cui fare l’inserimento della domanda e vedere la risposta.
      Tutta la parte dell’autenticazione è oggetto di sviluppi futuri.
      La parte principale non è ne il frontend ne il backend, ma quella che comunica con le 3 piattaforme e gestisce la risposta. La parte web può eessere una cosa anche molto base.
      Non serve una versione mobile e non provedere il caricamento di file (può essere buono ma non è obbligatorio). \\ \\
      
      \ni 2. Puo’ essere un’idea usare un prompt di sistema che aggiunge informazioni alla domanda inviata? (personalizzazione della domanda tramite l'uso di un prompt di sistema)
      & \ni E' un dettaglio, fuori dallo sviluppo principale. E' una cosa su cui concentrarsi in futuro. \\ \\
      
      \ni 3. Volete sicurezza e oscuramento dei dati sensibili?
      & \ni Oggetto di sviluppi successivi. La visibilità è solo interna all'azienda, non esterna (in azienda solo GitHub necessita di alcune autorizzazioni, ma poco anche lì). E' complesso da implementare. Da tenere per il futuro, il capitolato è già complesso senza. \\ \\
      
      \ni 4. Parlando di persistenza del database, quante domande per ogni chat devono essere salvate all'interno del db? (Nello specifico, possiamo mantenere nello storico chat un numero limitato di interazioni?)
      & \ni Salvate sia domande sia risposte. In futuro è possibile implementabile un feedback. Non c'è la questione dell'utenza, c'è un'unica utenza in cui viene mostrato tutto. Possibilità: storico di sessione (prossima sessione tutto vuoto) oppure "ultime 20 o 50 risposte" o "scroll" (scrolli in alto e si visualizzano le interazioni passate). Non perdere tempo sul frontend, meglio includere Slack e Telegram piuttosto. \\ \\
  \end{tabular}

  \vspace{0.5cm}

  {\Large Domande non funzionali}

  \vspace{0.5cm}

  \begin{tabular}{>{\justifying\arraybackslash}p{0.4\textwidth} >{\justifying\arraybackslash}p{0.6\textwidth}}
    \multicolumn{1}{c}{\textbf{Domande}} & \multicolumn{1}{c}{\textbf{Risposte}} \\ \\
    
    \ni 5. Ci potete fornire dei comportamenti tipo in modo da istruire il bot? (nello specifico, definiamo delle regole che il bot deve seguire)
    & \ni Le piattaforme sono incrociate tra loro. Su Confluence sono scritte le specifiche. Jira raccoglie i vari ticket. Su GitHub ci sono gli sviluppi di codice (con il commit che referenzia il ticket di Jira). Il chatbot deve essere istruito a collegare queste piattaforme tra loro. \\ \\
    
    \ni 6. Ci potreste dare APIkey di confluence, git, jira? allo stesso modo APIkey di OpenAI?
    & \ni Per Confluence e Jira c'è la versione gratuita, fino a 10 utenti. Quindi possiamo procurarci un nostro spazio Confluence e Jira. GitHub: possiamo usare anche una repo nostra personale. Per OpenAI, metteranno loro a disposizione il modo per utilizzarlo. Visto che noi avremo dei nostri spazi, facciamo prima così piuttosto che avere accesso a tutto il loro Confluence. Su Confluence noi saremmo utenti guest. Jira: in caso, noi comunque non avremo accesso a tutta la documentazione dei progetti, potremmo avere una configurazione ad hoc, un ambiente tutto nostro, per non accedere ai loro dati. \\ \\
    
    \ni 7. Per mostrarvi la documentazione tecnica, il bug reporting, il codice sorgente ci potreste  aprire su confluence,su git e su jira  una parte esclusiva per noi? (nello specifico, uso dei vostri sistemi per una comunicazione diretta)
    & \ni Non chiedono il tipo di report bug dove l'utente riporta un bug del bot. Per la documentazione tecnica, forse ci danno uno spazio su Confluence e Jira, ci diranno alla riunione introduttiva che faranno con i due gruppi che si saranno aggiudicati il progetto. Potenzialmente possono loro aprire la repo sui loro spazi e direttamente farci usare la sezione "Issues" di GitHub. Loro scrivono i bug, noi li risolviamo. \\ \\
    
    \ni 8. Ci potreste fornire degli schemi di come avete organizzato la documentazione su confluence, il codice su git e quanto specifico è il vostro sistema di ticket su jira?
    & \ni Risposta 8: Confluence è simile a documenti Markdown, è come un Drive. Del resto, ne si parla in fase introduttiva. I ticket su Jira hanno nome (5-6 parole), richiesta, descrizione e allegati. Buddybot non deve gestire allegati, deve solo leggere la descrizione. In Jira, hanno una repo di backend (typescript) e una di frontend (JavaScript). Nulla da dire di particolare sulla struttura Jira, dunque dovremo allenare il bot ad accedere ad una normale board di Jira. \\ \\
\end{tabular}

\begin{tabular}{>{\justifying\arraybackslash}p{0.4\textwidth} >{\justifying\arraybackslash}p{0.6\textwidth}}
    \ni 9. Volete per forza un'architettura in microservizi? (Nello specifico, si possono usare software monolitici per risparmiare risorse?)
    & \ni Piena libertà sulla metodologia. Riconoscono che è più rapido fare in modo alternativo, quindi ce lo accettano. \\ \\
\end{tabular}

\endgroup