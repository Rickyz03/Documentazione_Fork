\section{Diario della riunione}

\begin{itemize}
    \item Analizzata la presentazione del capitolato \emph{LLM: Assistente virtuale} di \emph{Ergon}, e raccolte domande e dubbi nel Foglio Google consultabile \ulhref{https://docs.google.com/spreadsheets/d/1W68ouX5T0ck2tDQm2uQCv7EYVG3xXCtpChtITe8GQg0/edit?gid=0}{qui}.
    
    \item Analizzata la documentazione del capitolato \emph{NearYou - Smart custom advertising platform} di \emph{SyncLab}, e abbiamo deciso di non procedere con l'interesse al capitolato, perchè abbiamo valutato l'impegno come aldilà delle nostre possibilità.
    
    \item Contattata via mail l'azienda \emph{Ergon} per organizzare un incontro online in cui rispondere e discutere delle domande presenti nei rispettivi Fogli Google.

    \item Accettato l'utilizzo di \emph{Overleaf} e \emph{GitHub Pages} per gestire la documentazione del gruppo

    \item Preparata una bozza di piano per l'approccio all'incontro di lunedì alle 17:00 con l'azienda \emph{Azzurro Digitale}. 
    \begin{itemize}
        \renewcommand{\labelitemii}{--}
        \item Inizialmente presentare qual è la nostra idea, cioè che cosa abbiamo capito del progetto, per dimostrare che lo abbiamo analizzato per bene.
        \item Presentare via Zoom il Foglio Google con le domande che abbiamo in relazione all'azienda (le quali saranno divise in "funzionali" e "non funzionali") e scrivere in live le risposte.
        \item Applicare il metodo MoSCoW come tecnica di prioritizzazione dei requisiti del progetto, cioè fare domanda all'azienda per capire cosa ne pensano di ciascun requisito.
    \end{itemize}
\end{itemize}