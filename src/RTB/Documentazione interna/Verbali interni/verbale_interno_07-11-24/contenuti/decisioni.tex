% Intestazione
\fancyhead[L]{4 \hspace{0.2cm} Decisioni} % Testo a sinistra


\section{Decisioni}

Durante la riunione sono state prese le seguenti decisioni:

\vspace{0.5cm}

\begin{table}[htbp]
    \centering
    \rowcolors{2}{lightgray}{white}
    \begin{tabular}{|c|p{0.8\textwidth}|}
        \hline
        \rowcolor[gray]{0.75}
        \multicolumn{1}{|c|}{\textbf{Codice}} & \multicolumn{1}{|c|}{\textbf{Descrizione}}\\
        \hline
        VI 8.1 & È stato deciso di incominciare a scrivere i documenti per la RTB, in alcune parti fattibili e prestabilite, entro lunedì 11 novembre,
        dove svolgeremo una riunione interna nella quale prepareremo l'incontro con \emph{AzzurroDigitale} del giorno dopo \\
        \hline
        VI 8.2 & Sono state decise le metriche di qualità riguardanti la documentazione, cioè gli obiettivi di correttezza linguistica e leggibilità.
        In particolare, abbiamo stabilito per gli errori ortografici un valore accettabile di 5\% e un valore ottimale di 0\%, e per la leggibilità abbiamo
        stabilito di utilizzare l'Indice di Gulpease, con un valore accettabile di 40 e un valore ottimale di 60 \\
        \hline
        VI 8.3 & È stato deciso di utilizzare \emph{GitHub Projects} per la gestione del backlog, poichè esso è già incluso nel nostro ambiente di lavoro e ci
        permette di avere un'interfaccia unica per tutto il progetto \\
        \hline
        VI 8.4 & È stato deciso di utilizzare \emph{Fogli Google} per creare tutti i grafici e alcuni diagrammi legati al progetto, poichè lo strumento 
        facilita l'inserimento e la gestione di grandi quantità di dati, ed è inoltre già mediamente conosciuto dai membri del gruppo, a differenza ad 
        esempio di uno strumento come \emph{Jira} \\
        \hline
        VI 8.5 & È stato deciso di mantenere due repo separati, uno per la documentazione e uno per il codice, e di utilizzare \emph{GitHub Projects} per la
        gestione del backlog di entrambi i repo \\
        \hline
    \end{tabular}
    \end{table}
