% Intestazione
\fancyhead[L]{Ordine del giorno} % Testo a sinistra


\section{Ordine del giorno}

\begin{itemize}
    \item Item 1 : Da \ulhref{https://www.google.it}{qui} puoi visitare Google
    \item Item 2
    \item Item 3
\end{itemize}

\vspace{2cm}

\begingroup
\renewcommand{\ni}{\noindent}

{\Large Domande funzionali}

\vspace{0.5cm}

\begin{tabular}{>{\justifying\arraybackslash}p{0.4\textwidth} >{\justifying\arraybackslash}p{0.6\textwidth}}
    \multicolumn{1}{c}{\textbf{Domande}} & \multicolumn{1}{c}{\textbf{Risposte}} \\ \\
    \ni Questa è la domanda 1 & \ni Questa è la risposta alla domanda 1. Può essere lunga e si adatta alla larghezza della colonna senza problemi. \\ \\
    \ni Questa è la domanda 2 & \ni Questa è la risposta alla domanda 2. Anche qui si possono aggiungere dettagli lunghi senza problemi. \\ \\
    \ni Questa è la domanda 3 & \ni Questa è la risposta alla domanda 3. \\ \\
\end{tabular}

\endgroup